\section{Introduzione}
\subsection{Abstract}
Il progetto sviluppato propone la realizzazione di un sito che mostri le creazioni acquistabili presso la “Fioreria all’Arco”, fioreria della mamma del responsabile del gruppo. In un momento delicato come questo e in un mondo ormai così legato al virtuale, l’idea era di rendere sempre accessibile tutti i prodotti e i dettagli della fioreria, in modo tale da permettere ai clienti interessati di osservare le varie composizioni nelle varie occasioni e di avere la possibilità di vedere come poter contattare direttamente la negoziante o il negozio stesso.\\
Allo stesso tempo, è stato anche necessario creare un’interfaccia che permetta all'amministratore di inserire, aggiornare ed eliminare i contenuti in continuo aggiornamento dal sito. Di conseguenza è stata creata un’interfaccia che permetta all’amministratore di fare tutto ciò che è a lui necessario, senza aver bisogno di conoscere i linguaggi di programmazione che modellano il sito. \\
Sia il sito della fioreria che l’interfaccia sono state sviluppate secondo le specifiche direttive dell’amministratore.


\subsection{Utenti destinatari}
Il sito “Fioreria all’Arco” è destinato a tutti gli utenti che sono interessati a comprare composizioni floreali per ogni occasione e anche per chi vuole solamente soddisfare la sua curiosità e vedere quali tipi di creazioni vengono vendute in questa fioreria. 
Inoltre, all’interno del footer sono presenti i link per collegarsi alle pagine social, dove sarà possibile reperire informazioni sugli aggiornamenti della fioreria e sulle ultime novità che essa propone.\\
L’interfaccia admin dell’omonimo sito è invece destinata agli amministratori (non solo la negoziante ha il suo identificativo, ma anche i suoi figli ed eventuali colleghi che lavorano all’interno della fioreria) che possono gestire il contenuto del sito.


\subsection{Possibili sviluppi futuri}
Per il sito principale “Fioreria all’Arco”, un’idea per uno sviluppo futuro riguarda l’ambito della vendita online, quindi di un sito e-commerce.\\
L’interfaccia admin, invece, potrebbe svilupparsi ulteriormente aggiungendo maggiori funzionalità, come ad esempio la gestione di diversi utenti-editor da parte di un solo amministratore che possiede più permessi o la creazione di una chat virtuale con la quale poter rispondere alle richieste dei clienti direttamente da questa interfaccia. Per ora è possibile visualizzare tutte le richieste inoltrate tramite il form "contattaci" del sito e inviare una mail di risposta.
