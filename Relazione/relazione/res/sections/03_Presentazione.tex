\section{Presentazione}
Qualche parole riguardo l'estetica che si è deciso di usare (boh tipo il discorso delle variabili per i colori nel CSS), + il discorso del responsive
\subsection{Desktop}
Lorem ipsum dolor sit amet, consectetur adipiscing elit, sed do eiusmod tempor incididunt ut labore et dolore magna aliqua. 
\subsubsection{Cliente}
un piccolo screen
\subsubsection{Admin}
un piccolo screen

\subsection{Mobile}
Il layout mobile è principalmente improntato su una disposizione verticale dei vari elementi della pagina, distanziando tra loro gli elementi clickabili per evitare che l'user clicki involontariamente sull'elemento sbagliato. 
\subsubsection{Cliente}
Per la parte client l'header raggruppa il menù di navigazione in un menù ad hamburger, mentre il footer contiene solo gli orari di apertura e i social-media visto che le altre informazioni(email, numero di telefono e indirizzo) sono reperibili nella pagina dei Contatti. Inoltre per il corretto funzionamento del menù ad hamburger e del resto della pagina dei lavori specifici è stato utilizzato l'attributo \textit{!important} perché non era possibile fare in altri modi. Per le pagine di esposizione dei lavori, in particolare per la gallery, si è deciso di passare da un layout a 4 immagini per riga ad un layout a 2 immagini per riga per evitare che le foto fossero troppo piccole e troppo vicine tra di loro. 
un piccolo screen
\subsubsection{Admin}
Per la parte admin si è semplicemente scelto di posizionare il menù di navigazione in alto al posto di tenerlo fisso a sinistra.
un piccolo screen

\subsection{Print}
Lorem ipsum dolor sit amet, consectetur adipiscing elit, sed do eiusmod tempor incididunt ut labore et dolore magna aliqua. 
\subsubsection{Cliente}
un piccolo screen
\subsubsection{Admin}
un piccolo screen
