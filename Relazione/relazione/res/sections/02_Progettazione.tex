\section{Progettazione}
Nella fase iniziale, il gruppo ha deciso come strutturare il sito, sia dal per l'amministratore, che per l'utente visitatore, in base degli obiettivi prefissati e alle linee guida da seguire.\\
È stata definito lo schema generale del database, pensandolo nella maniera più mantenibile per sviluppi futuri. \\
Successivamente sono stati definiti i diversi layout per ogni pagina del sito, pensando ai diversi schermi di visualizzazione.\\Pianificando gli obiettivi, sono stati suddivisi i compiti tra i membri del gruppo.

\subsection{Classificazione degli utenti}
I tipi di utente sono:
\begin{enumerate}
\item \textbf{cliente}, \\ovvero l'utente interessato ai contenuti del sito, quindi:
\begin{itemize}
\item può visualizzare i lavori che la fioreria produce, anche nel dettaglio;
\item tramite la pagina contatti, può: 
\begin{itemize}
\item visualizzare tutte le informazioni identificative riguardo la fioreria;
\item contattare l'amministratore tramite un apposito form;
\end{itemize}  
\end{itemize}  
\item \textbf{admin}, \\ovvero l'utente interessato alla gestione dei contenuti del sito, quindi:
\begin{itemize}
\item può gestire le richieste che i clienti mandano tramite l'apposito form; 
\item può gestire le categorie di lavori:
\begin{itemize}
\item creare una nuova categoria di lavoro;
\item modificare i dati di una nuova categoria di lavoro già inserita;
\item eliminare una categoria di lavoro già inserita;
\end{itemize} 
\item può gestire gli orari del pubblico da mostrare;
\item può gestire gli altri amministratori del sito, e eventualmente aggiungerne uno;
\end{itemize}
Credenziali degli utenti admin inseriti:
\begin{center}
\begin{tabular}{|p{0.2\textwidth}|p{0.2\textwidth}|}
\hline
username          & password        \\
\hline
admin           & admin     \\
admin2         & admin2     \\
admin3         & admin3     \\
prova         & prova     \\
\hline
\end{tabular}
\end{center}
\end{enumerate}

\subsection{Database}
Viene ora mostrata la struttura del database del sito, il quale permetterà la gestione degli utenti admin, delle richieste inviate dai clienti, delle impostazioni e dei vari lavori. Ogni lavoro ha più di un'immagine, mentre un'immagine può appartenere ad un solo lavoro.\\C'è poi una tabella per la gestione delle richieste dei clienti, una per la gestione degli admin e una per le impostazioni, ovvero le informazioni tecniche dell'azienda.
\begin{center}
\includegraphics[scale = 0.5]{../latex/images/db.png}\\[1.5cm]
\end{center}

\subsection{Struttura dei contenuti}
Il sito si sviluppa nelle sue due parti, ovvero \textit{lato cliente} per i clienti interessati alla fioreria e ai prodotti che essa offre, e \textit{lato admin} per gli amministratori che si occupano della gestione dei contenuti del sito della fioreria. 

\subsubsection{Lato cliente}
Vengono ora descritte e argomentate le pagine che compongono il sito nel \textit{lato cliente}.\\\\
\textbf{Pagine}\\
qualcosa qualcosa qualcosa qualcosa	
	\begin{itemize}
		\item \textbf{Home} \\qualcosa qualcosa
		\item \textbf{Chi siamo}\\qualcosa qualcosa
		\item \textbf{Lavori}\\dire le pagine che lo compongono e cosa contengono
	 	\begin{itemize}
 			\item Matrimoni;\\	qualcosa qualcosa 
	 		\item Lauree;\\	qualcosa qualcosa 
 			\item Funerali;\\	qualcosa qualcosa
 			\item ...; 		 		
	 	\end{itemize}
	 	\item \textbf{Contatti}\\qualcosa qualcosa\\
 	\end{itemize}
\textbf{Parti fondamentali -> troviamoci un sinonimo}\\ 
	qualcosa qualcosa qualcosa
	\begin{itemize}
		\item \textbf{Header}\\qualcosa qualcosa
		\item \textbf{Breadcrumb}\\qualcosa qualcosa
		\item \textbf{Footer}\\qualcosa qualcosa
 	\end{itemize}
 		
\subsubsection{Lato Admin}
Come prima cosa è bene precisare che tutti gli admin inseriti hanno tutti gli stessi permessi sulla gestione di tutto il sistema. Questo è stato fatto perchè ad avere accesso all'interfaccia di amministrazione saranno un numero ristretto di fidate persone.\\
Ogni amministratore si collega alla pagina di login e si autentica con le credenziali personali. È stata anche implementata una variabile di \texttt{SESSION}, per poter evitare che terzi possano accedere alle pagine di amministrazione senza essersi autenticati.\\
Vengono ora descritte e argomentate le pagine che compongono il sito nel \textit{lato cliente}.\\\\
\textbf{Pagine}\\ Il layout desktop per l'amministratore presenta un menù laterale a destra con tutte le pagine dove esso si può muovere.
	\begin{itemize}
		\item \textbf{Dashboard};\\In questa pagina l'amministratore vede le ultime richieste inviate dai clienti e può modificare gli orari di apertura del negozio. Si è deciso di dare la possibilità all'amministratore di modificare solo gli orari, poichè il numero di telefono, l'email e l'indirizzo non sono dati che cambiano spesso, come invece lo sono gli orari di apertura. Uno sviluppo futuro potrebbe essere proprio quello di aggiungere uno spazio per la modifica anche di queste informazioni.
		\item \textbf{Gestione richieste};\\qualcosa qualcosa
		\item \textbf{Gestione categorie};\\dire le pagine che lo compongono e come funziona il tutto
	 	\begin{itemize}
 			\item Aggiungi categoria;\\qualcosa qualcosa
 			\item Modifica categoria;\\qualcosa qualcosa
 			\item Elimina categoria;\\qualcosa qualcosa
	 	\end{itemize}
 	\item \textbf{Gestione utenti}\\qualcosa qualcosa qualcosa
 	\item \textbf{Logout}\\Quando l'amministratore clicca su questa voce del menù, effettua il logout dalla piattaforma e viene reindirizzato alla pagina di login. La variabile di \texttt{SESSION} viene quindi chiusa e quindi se l'amministratore prova ad accedere ad una specifica pagina di amministrazione dopo aver fatto il logout, viene reindirizzato alla pagina di login per potersi autenticare.\\
 	\end{itemize}
\textbf{Parti fondamentali -> troviamoci un sinonimo} \\
\begin{itemize}
	\item \textbf{Breadcrumb}\\	qualcosa qualcosa qualcosa
	\item \textbf{Funzionamento PHP}\\	DbConnection ecc.
\end{itemize}